\documentclass[13pt,]{tufte-handout}

% ams
\usepackage{amssymb,amsmath}

\usepackage{ifxetex,ifluatex}
\usepackage{fixltx2e} % provides \textsubscript
\ifnum 0\ifxetex 1\fi\ifluatex 1\fi=0 % if pdftex
  \usepackage[T1]{fontenc}
  \usepackage[utf8]{inputenc}
\else % if luatex or xelatex
  \makeatletter
  \@ifpackageloaded{fontspec}{}{\usepackage{fontspec}}
  \makeatother
  \defaultfontfeatures{Ligatures=TeX,Scale=MatchLowercase}
  \makeatletter
  \@ifpackageloaded{soul}{
     \renewcommand\allcapsspacing[1]{{\addfontfeature{LetterSpace=15}#1}}
     \renewcommand\smallcapsspacing[1]{{\addfontfeature{LetterSpace=10}#1}}
   }{}
  \makeatother

\fi

% graphix
\usepackage{graphicx}
\setkeys{Gin}{width=\linewidth,totalheight=\textheight,keepaspectratio}

% booktabs
\usepackage{booktabs}

% url
\usepackage{url}

% hyperref
\usepackage{hyperref}

% units.
\usepackage{units}


\setcounter{secnumdepth}{-1}

% citations

% pandoc syntax highlighting

% longtable

% multiplecol
\usepackage{multicol}

% strikeout
\usepackage[normalem]{ulem}

% morefloats
\usepackage{morefloats}


% tightlist macro required by pandoc >= 1.14
\providecommand{\tightlist}{%
  \setlength{\itemsep}{0pt}\setlength{\parskip}{0pt}}

% title / author / date
\title{Marketing for Self-Made Experts}
\author{Philip Morgan}
\date{Oct 22, 2019 - v0.5}


\begin{document}

\maketitle

\begin{abstract}
\noindent What got you here won't get you there.

At some point, the tools that you begin using in your marketing --
direct response tools -- will interfere with the trust needed to
establish your authority as an expert. This short guidebook describes
this tension and how to navigate it.
\end{abstract}


{
\hypersetup{linkcolor=black}
\setcounter{tocdepth}{3}
\tableofcontents
}

\hypertarget{introduction}{%
\section{Introduction}\label{introduction}}

This is a common scenario:

A consultant, living on word of mouth and referrals, decides they need
to ``do marketing''. They're tired of the unpredictability and lack of
control.

There are two ways that consultant can connect and build trust with
prospects outside of referrals and word of mouth: direct response
marketing or brand marketing. I'd bet money that the consultant chooses
direct response marketing, and most times I'd win if someone bet against
me.

After reading this short guidebook, you'll know why not to bet against
me here, and you'll understand why choosing direct response marketing,
which is often chosen without knowledge of the downstream consequences,
is necessarily a short term, utilitarian choice that must later give way
to choosing brand marketing.

If our new-to-marketing consultant cultivates genuine expertise, at some
point the perceived value of their expertise will be undermined by their
use of direct response tools. They'll have chosen a marketing approach
that conflicts with the growth of their business, and they'll need to
change.

This book is not meant to help you ``do marketing''. It's meant to help
you think about how to connect and build trust with prospective clients
from the beginning to the end of a career where self-made expertise
drives both client outcomes and the consultant's increasing authority
and profitability.

If you're like most consultants delivering genuine expertise, you'll
probably start out using direct response marketing, gain momentum using
that mode of marketing, and then face a tension between continued
success and your chosen mode of marketing. I wrote this book so you can
navigate this path without crashing headlong into a frustrating tension
that saps your momentum for several years.

This book is for self-made experts. We're the ones who will need to
cross a chasm at some point from direct response to brand marketing.

\hypertarget{definitions}{%
\subsection{Definitions}\label{definitions}}

I'm going to help you fully understand brand and direct response
marketing shortly. Right now, let's start with some definitions:

\begin{itemize}
\tightlist
\item
  \textbf{Brand marketing}: Art with a logo on it.
\item
  \textbf{Direct response marketing}: A button or form with a funnel
  behind it.
\end{itemize}

These definitions are constructed in a playful way to make them
memorable and sticky, but they also express useful, serious ideas.

Let's see if these definitions help you with a quick quiz.

Q: Is email marketing brand or direct response marketing?

A: It can be either. It can be art with a logo on it. And it can -- and
very often is -- an opt-in form with a funnel behind it. Even the same
email list can function in both modes at different times.

Q: Is giving a talk at a conference brand or direct response marketing?

A: Again, it could be either. It could be an artful sharing of expertise
with a logo (your personal brand, in this case) woven throughout it. Or,
it could be designed to efficiently obtain contact information from a
segment of the audience and then feed that contact info into a digital
marketing funnel. Or, it could be both.

By the way, don't get thrown off by an overly-specific interpretation of
the word ``art'' when you think about brand marketing. I don't mean
visual art, music, poetry, etc. Brand marketing art is \emph{an
appealing or challenging gift to a culture}.

\hypertarget{onward}{%
\subsection{Onward}\label{onward}}

I have never felt like more words makes a book better. The rest of this
book will maintain the lean, efficient treatment of the topic at hand.
If you can read this book in one sitting, I'll be delighted.

Onward to a deeper understanding of our ``villain with a heart of gold''
here: direct response marketing.\# What is Direct Response Marketing?

Direct response marketing is a button or form with a funnel behind it.

It's useful to think about direct response marketing as a set of
concrete \emph{tools}, and also as a basket of less concrete ideas and
feelings, which I'll label the \emph{genre expectations} around direct
response marketing. We'll use that same lens to examine brand marketing.

\hypertarget{the-tools}{%
\subsection{The Tools}\label{the-tools}}

If you see any of these tools in use, you're seeing some form of direct
response marketing in action:

\hypertarget{a-clear-call-to-action}{%
\subsubsection{\texorpdfstring{A clear \emph{call to
action}}{A clear call to action}}\label{a-clear-call-to-action}}

A call to action is a request that someone take some form of action. It
can be as small as the ``Buy Now'' microcopy on a web page button, and
may often be as large as a paragraph of text.

Ex: ``To get the free bonus, send me your Amazon receipt showing that
you've pre-purchased 10 or more copies of my book.''

\hypertarget{a-low-friction-form-used-to-collect-signups-or-opt-ins.}{%
\subsubsection{\texorpdfstring{A low-friction \emph{form} used to
collect signups or
opt-ins.}{A low-friction form used to collect signups or opt-ins.}}\label{a-low-friction-form-used-to-collect-signups-or-opt-ins.}}

Forms are typically used in direct response marketing to give people the
option of joining an email list, signing up to receive a gated content
asset, or sign up for an event. The underlying purpose is to gather
information about that person, establish the beginnings of a 1:1
relationship, or find out where that person fits within a model of your
audience (are they ready to buy or somewhere prior to that in their
journey?).

Ex: ``To receive this free email course, tell me what email address to
send it to.''

\hypertarget{a-gated-content-asset}{%
\subsubsection{\texorpdfstring{A \emph{gated content
asset}}{A gated content asset}}\label{a-gated-content-asset}}

A content asset is something like a white paper, electronic book, or
something similar. The ``gate'' in front of these types of content
assets is an opt-in form, or a form that progressively builds a profile
of that person.

Ex: ``To download this white paper, please tell us about your company
demographics and needs.''

\hypertarget{a-low-priced-product-used-to-measure-buying-intent-and-collect-contact-information.}{%
\subsubsection{\texorpdfstring{A \emph{low-priced product} used to
measure buying intent and collect contact
information.}{A low-priced product used to measure buying intent and collect contact information.}}\label{a-low-priced-product-used-to-measure-buying-intent-and-collect-contact-information.}}

Price is always a relative thing, and so ``low-priced product'' might
mean anything from an e-book priced the same as a fast food restaurant
meal all the way to a \$2,000 seminar that delivers a taste of what a
\$200,000 consulting engagement would be like. The role of low-priced
products in direct response marketing is generally either 1) to
establish a financial relationship or 2) to collect contact information
of more potentially serious prospects.

Ex: ``Buy this \$7 ebook and learn how to get rid of painful corns and
callouses.''

\hypertarget{an-event-used-to-collect-contact-information-that-will-later-be-used-to-promote-something.}{%
\subsubsection{\texorpdfstring{An \emph{event} used to collect contact
information that will later be used to promote
something.}{An event used to collect contact information that will later be used to promote something.}}\label{an-event-used-to-collect-contact-information-that-will-later-be-used-to-promote-something.}}

We seem to be in the waning hours of the extreme popularity of webinars,
but for quite a while we've seen webinars -- an online event -- used to
deliver a small amount of value while mainly working to collect contact
information that will later be used to promote something.

Ex: ``Attend this free webinar and learn how to price your services more
profitably.''

\hypertarget{long-form-sales-copy}{%
\subsubsection{\texorpdfstring{\emph{Long form} sales
copy}{Long form sales copy}}\label{long-form-sales-copy}}

Direct response marketing is often trying to eliminate or reduce the
role of high-touch sales. If this form of marketing can get something
sold without 1:1 human intervention, it's seen as a win.

That's why direct response marketing often makes use of long form sales
copy. Direct response marketing is generally operating outside the
infrastructure of established distribution channels and institutions, so
the job of selling falls to the marketer, and direct response marketers
want to do this selling efficiently at scale. To efficiently push
prospects from unaware to aware to prospect to customer, long form sales
copy is used.

\hypertarget{a-sequence-of-emails-that-describes-some-pain-or-problem-spends-time-vivifying-that-painproblem-and-pitches-a-solution-to-that-painproblem.}{%
\subsubsection{\texorpdfstring{A \emph{sequence of emails} that
describes some pain or problem, spends time vivifying that pain/problem,
and pitches a solution to that
pain/problem.}{A sequence of emails that describes some pain or problem, spends time vivifying that pain/problem, and pitches a solution to that pain/problem.}}\label{a-sequence-of-emails-that-describes-some-pain-or-problem-spends-time-vivifying-that-painproblem-and-pitches-a-solution-to-that-painproblem.}}

If you take long form sales copy, which is generally arranged vertically
in a single very long column on a single sales page and divide it up
into multiple emails, you have an email sequence that has the job of
selling something, or pushing a prospect closer to taking some desired
action.

\hypertarget{moneyback-guarantees}{%
\subsubsection{\texorpdfstring{\emph{Moneyback
guarantees}}{Moneyback guarantees}}\label{moneyback-guarantees}}

Lots of products and services include a moneyback guarantee. When you
see a moneyback guarantee listed as a \emph{primary feature} of the
product/services, you're probably seeing direct response marketing in
action.

\hypertarget{testimonials}{%
\subsubsection{\texorpdfstring{\emph{Testimonials}}{Testimonials}}\label{testimonials}}

A client or customer going on record with a description of, endorsement
of, or praise for your product/services is a testimonial. Because direct
response marketing seeks to build trust in a rapid and scalable manner,
it tends to make use of testimonials. Testimonials in direct response
marketing are used the way case studies are used in other forms of
marketing. ``See, this stuff works!''

\hypertarget{engineered-pricing}{%
\subsubsection{\texorpdfstring{\emph{Engineered
pricing}}{Engineered pricing}}\label{engineered-pricing}}

Direct response marketing, when used to sell something, often uses
``engineered pricing''. To an extend, all pricing is engineered to
achieve some purpose: volume, revenue, or profitability. But direct
response marketing often uses somewhat elaborate pricing models
involving price tiers, bonuses, and so-called choice architecture or
buying psychology. Additionally, direct response marketing will often
use a focus on some painful problem and artificial scarcity to justify
super-premium pricing.

If you see a price schedule with 3 or more tiers, often represented as
columns in a table, then you are either seeing the price for a SaaS or a
product/service with engineered pricing.

\begin{center}\rule{0.5\linewidth}{\linethickness}\end{center}

OK, that's our list of the primary tools of direct response marketing.
Onward to the genre expectations.

\hypertarget{genre-expectations}{%
\subsection{Genre expectations}\label{genre-expectations}}

If we think of direct response marketing as a \emph{genre}, then there
are -- as with any genre -- certain expectations about how the form
works:

\hypertarget{connection}{%
\subsubsection{1:1 Connection}\label{connection}}

The ethos of direct response marketing is creating a 1:1 connection with
leads from the very first conversion event, combined with collecting
enough information to enable highly efficient marketing to that lead.

\hypertarget{data}{%
\subsubsection{Data}\label{data}}

Direct response marketing will intentionally or accidentally collect
data -- the more individualized and complete the better -- about those
being marketing to. This is why the Internet, at its most fundamental,
is inherently a direct response medium.

\hypertarget{goal}{%
\subsubsection{Goal}\label{goal}}

The goal of direct response marketing is to \emph{get a response} from
those we are marketing to. This is the most fundamental, defining aspect
of the genre. The ``response'' is not necessarily a sale. It might be
some other kind of action: filling out a form, clicking a button,
opening an email, attending an event, joining a waiting list, or the
like.

This idea of getting a response is elaborated into the model of a
``digital marketing funnel''. A funnel attempts to use a variety of
techniques \emph{over time} to push people from unaware to aware to
prospect to customer. At each stage of the funnel, some may not move
forward or may leave the funnel. That's why the model of a funnel --
moving from the broad mouth to the narrow neck -- is used rather than
the model of a straight pipe.

\hypertarget{problems-are-markets}{%
\subsubsection{Problems are markets}\label{problems-are-markets}}

Finally -- critically! -- direct response marketing is focused on
\emph{problems}. This is not always the case, but very often the direct
response marketing relationship begins with a conversion event (someone
filling out a form, for example) that's focused on helping that person
solve a problem.

\hypertarget{tone}{%
\subsection{Tone}\label{tone}}

The genre of heavy metal music prepares you to hear distorted electric
guitars, heavy drumming, and screamed vocals. There is a fun sub-genre
of novelty heavy metal music. In that sub-genre, you have groups like
Hayseed Dixie playing AC-DC songs on guitar, mandolin, banjo, and fiddle
in a bluegrass style. The difference is not the notes being played, it's
the entire \emph{tone} of the resulting music.

A big part of the genre expectations around direct response marketing is
the tone used in the marketing. I think of these like the ``tone knobs''
on an electric guitar amplifier.

There are a million ways the tone of direct response marketing can be
good, crappy, or used as a lever to achieve the goals of the marketing,
which is really one singular goal: to get a desired response -- a button
click, a form filled out, or some other action taken. So I won't be able
to, nor should I, cover every single ``tone knob'' you can manipulate in
direct response marketing. That said, many of you -- me included! --
benefit from using a certain amount of direct response marketing in our
businesses. {[}\^{}And critically, our clients also benefit from us
using direct response marketing because it helps efficiently match their
problems with our solution. If we were the only side of the relationship
that benefits but it harms our clients, then of course we avoid the
temptation of benefitting at their expense.{]} Direct response is, in
many contexts, a useful tool. Since it's a mere tool, it's all about
\emph{how we use it}. And tone is a big part of that ``how we use it''
question. So let's talk about some of the more common direct response
tone knobs.

\hypertarget{urgency}{%
\subsubsection{Urgency}\label{urgency}}

Direct response marketing often tries to manufacture or amplify urgency
around the recipient taking some kind of action. Again, the action could
be a small next step, like opting in for something, or clicking a
button.

Sometimes this urgency comes merely from the tone. It's not so much
\emph{what's} being said as \emph{how} it's being said. A comparison of
2 made-up examples will help:

\begin{enumerate}
\def\labelenumi{\arabic{enumi}.}
\tightlist
\item
  The room for this event seats 100, and 30 seats remain at the time of
  writing.
\item
  Seats for this event are going like hotcakes! We've only got 30\% of
  the space left, and at this rate those seats won't last long. Claim
  yours now before it's too late.
\end{enumerate}

Those both convey similar information, but in very different ways. The
difference is how the latter uses a more intense tone to manufacture
urgency.

What counts as ``intense'' is audience-dependent.

Urgency can be manufactured via real or artificial scarcity. Sometimes
artificial scarcity, like the above example event with 30\% of seats
available, comes from obfuscating details that, if made known, would
relax the feeling of urgency. For example, if some e-commerce site shows
an available stock of 2 items, what does that really mean for me? Does
that mean they'll possibly be out of stock for weeks if I don't order
now? Or when they're out of stock, does their supplier easily have more
for them in a day or two? Or are they drop-shipping and the whole idea
of what's ``in stock'' is bullshit? By not revealing this information,
I'm nudged towards fear of missing out and behave differently as a
result.

Amplifying the consequences of inaction is another way of manufacturing
urgency. Calling anything the ``opportunity of a lifetime'' rings hollow
and false when -- for many of us especially in the global West -- we are
bombarded with opportunity from every side.

\hypertarget{pain}{%
\subsubsection{Pain}\label{pain}}

A lot of direct response marketing focuses on the \emph{pain} of the
problem the marketer purports to solve. The pain of doing nothing. The
pain of the status quo.

How exactly that pain is described contributes to the tone.

If you look on the Internet for copywriting advice, most of that advice
will be framed in the context of direct response marketing, and
therefore most of it will talk about ``agitating the pain''. By this,
those advice-givers mean for you to remind your reader about the pain
you purport to solve, use vivid imagery to do this reminding, and find
creative ways to bring that pain to life through your writing.

Of course, copywriters gonna copywrite, and some will greatly exaggerate
pain points. The visual apogee of this can be seen in this supercut of
ridiculous exaggerated pain from TV infomercials:

\url{https://youtu.be/qM4zMofsI7w}

One of the ways pain gets amplified in direct response marketing is to
focus on the \emph{emotional} aspects of the pain. Direct response
copywriters will tell you that \emph{all decisions} are emotional first,
backed up by a post hoc logical process. That may be true. {[}\^{}Or
this old saw about emotional decisions may not be true. I can't know
without doing my own due diligence. Every time I get exposed to a
supposedly science-based, surprising ``fact'' that can be expressed with
no ambiguity or subtlety in a headline, I get real suspicious that the
underlying science is flimsy or the translation to a headline has done
violence to the nuance of the underlying science.{]}

The problem is not how decisions are \emph{actually} made, the problem
is how we believe we make decisions. And if your audience fancies
themselves to be logical thinkers, a heavily emotional tone in direct
response marketing almost certainly will be problematic.

\hypertarget{curiosity}{%
\subsubsection{Curiosity}\label{curiosity}}

Sometimes you'll see direct response marketing make use of curiosity.
The apotheosis of curiosity in direct response marketing is the
clickbait headline. Some examples, courtesy of
\href{https://www.makeuseof.com/tag/buzzworthy-5-clickbait-headlines-guaranteed-annoy-ask-results/}{this
article}:

\begin{itemize}
\tightlist
\item
  The Hot New Phone Everybody Is Talking About
\item
  You Won't Believe What Happened Next
\item
  Watch This Video To Discover The True Meaning Of Life
\item
  See How One Man Made \$\$\$ In His Bedroom
\item
  Health Insurance Companies HATE This New Trick
\end{itemize}

This kind of attempt at increasing curiosity is based on leaving out a
critical detail to make something more interesting than it really is.
Once you start noticing this technique, you see it absolutely everywhere
on the Internet.

Direct response marketers let themselves off the hook for using
curiosity to waste~people's time and money by considering the
satisfaction of the curiosity -- rather than an actual benefit to
someone's business -- as a form of value. ``Clicking/buying will let
them satisfy their curiosity, which is a benefit to them, even if the
book wasn't all that great.'' Sure, for a \$10 book that's not a crime.
For something more costly I don't think that's sufficient justification
for the misuse of curiosity.

It's also possible -- easy, in fact! -- to phrase a curiosity-inducing
sentence or paragraph in a way that overstates the benefit in question.
These kind of obvious overstatements might be effective on rubes, but
more critical prospects will have a negative reaction.

\hypertarget{benefits}{%
\subsubsection{Benefits}\label{benefits}}

The final ``tone control'' relates to how benefits are described.

Are they described accurately, along with reasonable context about
what's required to actually achieve those benefits and who is likely to
achieve them? Or are they exaggerated or articulated in a context-free
way?

Are the benefits described in factual or emotional language? Is an extra
20\% in revenue an extra 20\% in revenue, or is it a life-changing
financial opportunity that will, to
\href{https://genius.com/Tom-waits-step-right-up-lyrics}{quote Tom
Waits}, ``change your shorts, change your life, change your life''?

I've seen firsthand a situation where someone used direct response
marketing with the benefits tone control cranked way up who then
\emph{blamed the customer} if the customer failed to realize the
promised benefits. This is lazy and unethical behavior.

I definitely understand the impulse to \emph{simplify} marketing
messages. Done correctly, this helps both parties. It maximizes the
chances that a message will penetrate and land in a noisy, somewhat
chaotic communication environment.

Done incorrectly, simplification of a direct response marketing message
leads to overstating benefits and stripping out too much context,
leaving the marketing with an unrealistic promised benefit at its core.

\hypertarget{persuasion}{%
\subsection{Persuasion}\label{persuasion}}

These direct response ``tone knobs'' all relate to one idea which, in
the world of copywriting, is referred to as ``persuasion''.

There are just soooo many ways persuasion can be used well or slightly
misused by accident or outright intentionally abused.

Different outcomes here are based on 1) whether you are placing your
client's best interest equal to or ahead of yours, 2) whether you
understand the kind of tone that your clients will and won't accept in
marketing, and finally 3) the time horizon over which you are working
towards impact.

My guidance on tone can be boiled down to this simple test, for which I
have Jonathan Stark to thank: ``Would you do it that way if your client
was a beloved family member?''

\hypertarget{summary}{%
\subsection{Summary}\label{summary}}

You'll probably notice that much of the marketing you've seen on the
Internet uses the tools and hews to the genre expectations of direct
response marketing. That's because the Internet is fundamentally a
direct response medium. What occurs most naturally when connecting and
building trust with prospects online is collecting data about them and
using the multitude of tools the Internet offers to push them through a
funnel.

As a high-leverage, low-cost bootstrapping tool for the self-made
expert, direct response marketing is a godsend.

As you'll begin to see, direct response marketing works differently than
brand marketing, and can start to create more problems than it solves
for the evolving self-made expert.\# What Is Brand Marketing?

Brand marketing is art with a logo on it. Not the kind of art you'd find
in a gallery. Instead, brand marketing is the kind of art that is a gift
or a challenge to a specific culture.

A culture is a group of people with a shared identity. Seth Godin might
call them a tribe.

We're going to look at brand marketing through the lens of tools and
genre expectations. Like all art -- brand marketing is as much about the
feel and nuance of how it's done as it is about specific objective
qualities, so you may find my description of brand marketing less crisp
and objective-sounding. This is a domain of subtleties, so bear with me.

\hypertarget{what-is-brand-marketing}{%
\subsection{What is brand marketing?}\label{what-is-brand-marketing}}

Especially for consultants like us, what the heck actually \emph{is}
brand marketing?

If you've ever been through an airport terminal -- at least here in the
US -- you've certainly seen brand marketing from a big company like
Accenture. You've seen those trying-really-hard-to-be-interesting
posters.

The ones this guy is making fun of here:

\url{https://twitter.com/tomfgoodwin/status/1173942057074905088}

Or if you watched TV \emph{at all} in the 80's, you probably saw this
campaign, from the US Army:

\url{https://youtu.be/ms9pxvEbILs}

And most of us have seen the archetypical form of brand marketing,
Superbowl commercials:

\url{https://youtu.be/Q51iAWN1oY8}

None of this stuff is what brand marketing for small consultancies looks
like. But it \emph{is} all art with a logo on it.

Brand marketing for small-size consultancies is \emph{also} art with a
logo on it. It is expensive relative to direct response marketing. It is
potentially inefficient relative to direct response marketing. But it
does not represent an expense on the scale of a Superbowl commercial.
And -- critically -- brand marketing is not in tension with the
expectations around expertise.

\hypertarget{tools}{%
\subsection{Tools}\label{tools}}

As with direct response marketing, brand marketing tends to make use of
a certain set of tools. As a quick reminder, the tools of direct
response marketing are: Clear calls to action, forms, gated content
assets, low-priced product(s) as segmentation mechanism, events as a
front-end to promoting something, email sequences, long-form sales copy,
moneyback guarantees, testimonials, and engineered pricing.

Here are the tools of brand marketing:

\hypertarget{gifts}{%
\subsubsection{Gifts}\label{gifts}}

Have any of these lodged in your memory?

\begin{itemize}
\tightlist
\item
  A remarkable TED talk.
\item
  A commercial, some part of which became conversational currency used
  among your friends (the \href{https://youtu.be/JJmqCKtJnxM}{Budweiser
  Whassup? campaign} comes to mind here -- and skimming the
  \href{https://en.wikipedia.org/wiki/Whassup\%3F}{``In culture''
  section of the Wikipedia article} on that campaign is fascinating and
  relevant).
\item
  Something else you didn't pay for, but created memorable value for
  you.
\end{itemize}

This is how brand marketing functions as a gift. It gives more than it
takes.

Direct response marketing ``gifts'' -- things like lead magnets and
content upgrades and ``free bonuses'' bundled with info products -- are
not given as freely. There's almost always the requirement of an email
address in return, with the implication that you will be marketed to
following providing your email address. Calling this arrangement a gift
is pretty disingenuous.

Brand marketing, on the other hand, sometimes gives what could truly be
thought of as gifts. Not always, of course. Some brand marketing efforts
are not very good gifts. And some are just sloppy efforts at selling
something under the guise of making a gift for the culture. But brand
marketing generally trades in the currency of gifts.

In \emph{our} world, that of the small consultant, brand marketing gifts
can look like the following:

\begin{itemize}
\tightlist
\item
  A talk where you share generously with no expectation of getting
  business directly from the talk
\item
  Something like Basecamp's \href{https://basecamp.com/books}{free
  books}
\item
  Any book, in fact, where the ROI is dramatic
\item
  A very good \href{https://2bobs.com}{podcast series}
\item
  An email list like Corey Quinn's
  \href{https://www.lastweekinaws.com}{Last Week in AWS}
\end{itemize}

A truly good gift is one we're delighted with because the giver combined
their affection for us with creativity. This is true in brand marketing,
and in life.

\hypertarget{impactful-experiences}{%
\subsubsection{Impactful experiences}\label{impactful-experiences}}

One way in particular that you can give a gift to the group you serve is
through an impactful experience. In fact, it's the \emph{relevance} and
\emph{impactful nature} of gifts that make them valuable in the first
place. Small, forgettable brand marketing gifts are small and
forgettable because they weren't very relevant or impactful.

Here's an impactful experience from my life: reading Blair Enns'
\emph{The Win Without Pitching Manifesto}. At the time Blair published
it as a series of pages on his website -- for free -- in addition to
letting you pay for more convenient packagings of the book, so at the
time I paid nothing for it. Yet, it totally rewired how I thought about
certain things and inspired me to make some courageous choices. For
Blair, publishing it was at least partially brand marketing work, and
for me, reading it was an impactful experience.

\hypertarget{reach-repetition}{%
\subsubsection{Reach \& repetition}\label{reach-repetition}}

Brand marketing will seek to saturate our awareness. This happens by
maximizing reach and making use of repetition.

In using the tools of reach and repetition, brand marketing may
sacrifice focus. Trading away focus to gain reach would almost never
happen (intentionally, anyway) with a direct response marketing
campaign. In the direct response world, reaching ``unresponsive''
prospects would be seen as wasteful, and it would reduce some short-term
oriented KPI.

A quick example that returns us to that tweet I showed at the top of the
chapter: imagine that airlines made a practice of trying to identify
passengers that are flying for business purposes. Maybe they segment
their customer list, and look for those flying business class and/or
those whose ticket seems to have been purchased by a business, and
identify those folks as likely business travelers. (This doesn't sound
like a farfetched a use of ``data exhaust'', does it?)

Accenture wants to reach business travelers. They have a choice. a) They
can buy display ad space to have their brand marketing ads hung on the
walls of airports. These posters and murals will certainly be seen by
some business travelers and a lot of non-business travelers. Or, b) they
can rent a list of business travelers from the airlines and directly
contact those people, likely reaching a much higher percentage of
business travelers. This is brand vs.~direct response marketing.
{[}\^{}Although I'd argue the posters I've seen on airport walls aren't
much of a gift and so aren't very good brand marketing.{]}

This is an interesting example because not every person who walks
through an airport could possibly be an Accenture client. So by going
with the posters rather than the rented list, Accenture is trading focus
for reach. With many brands (shoe brands, for example), this trade makes
sense. Every human is a potential shoe brand customer, so every human
having positive feelings about Nike's products makes some kind of sense
for Nike.

From a direct response marketing perspective, this tradeoff is insane.

But focus and reach and efficiency are not all that matter here.
\emph{Signals} matter too. Part of the signal Accenture sends by having
posters and murals in airports is: ``we're a well-known consulting
company. If you happen to need one of those, you might contact us.''
This is the functional part of brand marketing's signaling.

The rest of the signal they're sending is: ``we're doing so freaking
well that we can afford to be `wasteful' with our advertising budget. If
you happen to need a consulting company, you might want to contact the
one that is so big and profitable that they can spend advertising
dollars wastefully.'' This \emph{status signalling}. This is the same
reason some services firms need an expensive office. It ain't because
the fancy office makes the work better!

I believe but can't prove that the ``no one got fired for hiring IBM''
mindset applies even at the small end of the company size spectrum. So
when our prospective clients see us using marketing that is somewhat
``wasteful'' (ex: taking time away from client work, getting on a plane
and flying to speak at an event to share an impactful message), we send
the same status signal Accenture is sending with their airport
advertising and certain firms are sending with their expensive office
leases.

\hypertarget{broadcast-platforms}{%
\subsubsection{Broadcast platforms}\label{broadcast-platforms}}

Brand marketing often uses broadcast means of delivering content. In the
old media world, this would be television, radio, and mass market print
media like major publications and billboards. In the new media world,
this is un-targeted or minimally targeted media like YouTube videos that
are discovered through search or social sharing, talks at events, online
publications, and so forth.

To translate this to the small consultant word, consider this: you're
going to give a talk at a conference. Who \emph{exactly} will be there?
Will any of them have the exact problem you help clients with?

You can't know this with any certainty ahead of time. So your talk at
that conference is somewhat ``un-targeted'' compared to direct response
marketing.

Brand marketing also tends to integrate with existing institutions. Just
like Salesforce or iOS are platforms with public APIs third-party
developers can use to build apps on top of, certain cultural
institutions function the same way in brand marketing.

Let's say I want to get an idea in front of marketers. I have a choice
of several institutions I can integrate with by giving a conference
talk. One is the Inbound conference that HubSpot runs every year. I
don't have to do the work of gathering \textasciitilde{}25k people. The
institution does that for me. I don't have to do the work of earning
those attendees' trust. The institution does that for me. I integrate
with the institution by knowing its norms, understanding what its
audience might want, and using that understanding to tailor my talk to
fit.

\hypertarget{genre-expectations-1}{%
\subsection{Genre expectations}\label{genre-expectations-1}}

Just like direct response marketing has a sort of genre ``feel'' to it,
so does brand marketing.

\hypertarget{generosity}{%
\subsubsection{Generosity}\label{generosity}}

Brand marketing works best when it's done in a spirit of generosity.
This is not unfettered generosity. There's still the background
expectation that brand marketing contributes to a business outcome.

But at the same time, brand marketing exudes a feeling of ``Company X is
doing so well that it can afford to give back without needing an
immediate return, and so the marketing that Company X engages in can be
less mercenary and more generous and long-term in nature.''

\hypertarget{connection-or-contribution-to-the-culture}{%
\subsubsection{Connection or contribution to the
culture}\label{connection-or-contribution-to-the-culture}}

Brand marketing is culturally aware. That's how it gives gifts! It's
aware of what would and would not be valued by the culture, and so quite
often the gifts that brand marketing gives are connected to what's
happening \emph{now} in the culture. That's what makes those gifts
relevant.

As of this writing in late 2019, a conference talk that makes sense of
how the tech world could constructively handle events like those
involving Epstein, the MIT Media Lab, and Richard Stallman would be a
culturally-aware gift. If the talk came from a tech-focused consulting
or PR firm, we'd understand they probably have a business motive in
giving that talk, but we'd experience the insight and sense-making in
the talk as an act of generosity, and one that is aware of what is
troubling or threatening to the culture today. This cultural awareness
increases the value and impact of the gift.

Direct response marketing tends to be less culturally aware. As
well-known direct response marketer Gary Bencivenga put it: problems are
markets.

If the problem defines a culture, then a focus on a problem might be
culturally aware. However, many problems are not cultures.

Alcohol addiction has created a culture of recovery, embodied in the
institution of AA. AA is problem-as-culture. But is there a similar
cultural institution around addition to nicotine? Maybe, but not that
I'm aware of. I've never heard of Smokers Anonymous.

Many problems have no institutionalized culture, and therefore no
cultural platform or institutions for brand marketing to integrate with.

\hypertarget{aspirational-focus}{%
\subsubsection{Aspirational focus}\label{aspirational-focus}}

Brand marketing often has an aspirational focus. The US Army's ``Be all
you can be'' campaign is a clear example of this.

Just imagine how direct response marketing would attempt to reach the
goal of recruiting more people to volunteer for service in the Army.
Direct response marketing might:

\begin{itemize}
\tightlist
\item
  Focus on how expensive college education is, and how service in the
  Army can help pay for a college education.
\item
  Focus on how many poor or rural areas in the US lack job prospects,
  and how service in the Army can lead to acquisition of marketable
  skills.
\item
  Focus on how socially isolating life can be, and how service in the
  Army leads to lifelong friendships.
\end{itemize}

The ``Be all you can be'' campaign \emph{hints at} some of these issues,
but it does so in the larger context of an aspirational, uplifting
message about personal achievement and transcending your limitations.

Brand marketing tends to shift the focus away from problems and towards
aspirations or solutions.

\hypertarget{action-flows-from-affinity-for-the-brand-or-importance-of-the-message}{%
\subsubsection{Action flows from affinity for the brand or importance of
the
message}\label{action-flows-from-affinity-for-the-brand-or-importance-of-the-message}}

Brand marketing will often \emph{not} make use of calls to action (CTAs)
at all, and will generally avoid forceful or overly-specific CTAs.

CTAs may be present (ex: the ``Be all you can be'' campaign has a phone
number at the end), but they are generally quite ``soft'' or ``quiet''
in tone. Here's what a soft or quiet CTA is \emph{not}:

\includegraphics{https://pmc-dropshare.s3-us-west-1.amazonaws.com/Photo-2019-10-06-11-01.jpg}

So how does a brand marketing message or campaign lead to the economic
outcome we desire without using strong CTAs? In other words, how does it
lead to the right people buying our services?

The answer to that question also partially answers this question: ``when
does a small consultancy transition away from direct response marketing
towards brand marketing?'' The answer: when you have something so
important and relevant to say that listeners are naturally incentivized
to take action \emph{on their own}, without a forceful CTA.

In other words, brand marketing doesn't require forceful CTAs because
the CTA is implied in some way:

\begin{enumerate}
\def\labelenumi{\arabic{enumi}.}
\tightlist
\item
  The thing I'm talking about here is so important and relevant and my
  expertise is so obvious that you can ``connect the dots'', which means
  you going to my website without me telling you to, looking up my email
  address without me telling you to, and contacting me without me asking
  you to.
\item
  The thing I'm talking about here was already on your mind, and you
  were probably already searching for a solution, so I've just inserted
  myself into an established process you have for finding a consultant
  to work with, and the rest is up to that process.
\end{enumerate}

CTAs may be present in brand marketing, but they won't be forceful.
They'll tend to be placed elsewhere (closer to the point of sale, so to
speak), not in the main body of the gift or performance.

\hypertarget{time-horizon}{%
\subsubsection{Time horizon}\label{time-horizon}}

Another reason why brand marketing will often not have a CTA, or have
only a soft, general CTA: the time horizon over which the marketing is
expected to work is much longer than direct response marketing. Years;
not weeks or months.

Over that longer time horizon, you may have produced a portfolio of
gifts for those you are trying to reach, and the ``residue'' -- repeated
exposure over time -- of that portfolio has created brand power. That
brand power comes from you or your business name being associated with
solving some problem, or creating some desired state.

This time horizon thing is also why direct response marketing can feel
so high-pressure compared to direct response marketing. It is! The
results of a direct response campaign are expected in weeks or months
rather than years, and one way to achieve that goal is to\ldots{} apply
pressure.

\hypertarget{bootstrapping}{%
\subsection{Bootstrapping}\label{bootstrapping}}

It's unlikely you can use brand marketing earlier on in your career as a
self-made expert. Here's why.

If you're a self-made expert, your transition from doing implementation
work to advisory work almost certainly involves a bootstrapping phase
when you are working to earn credibility and access to advisory
opportunities. Your reputation falls short of your appetite for advisory
work and so it's likely the profitability you'd need to invest in brand
marketing simply isn't present in your business (yet). You won't be able
to bear the expense of brand marketing. For example, travel to speak at
multiple conferences each year may be financially out of reach, while
reaching prospects through podcast guesting is not.

Brand marketing's avoidance of hard CTAs, avoidance of pressure (which
signals need), and usual avoidance of a short-term focus on problems
causes it to work more slowly. During a bootstrapping phase, you may not
be able to afford to invest in this kind of lead time.

Finally, brand marketing tends to rely on institutions to help you
distribute your message. These institutions earn trust by carefully
curating who they put in front of their audience. As an outsider, you
may have a hard time getting institutional acceptance or support. You're
an unknown quantity. Furthermore, your message may run counter to what
the institution wants to expose its audience to.

At some point, reputation becomes a ``trojan horse'' for a disruptive
message, or a message in search of an audience. But before you have that
reputation, what institutions will give you a platform? If none, then
you need direct response marketing rather than brand marketing, at least
until you bootstrap your reputation.

\hypertarget{summary-1}{%
\subsection{Summary}\label{summary-1}}

At some point in the past, I remember mocking advertising with no clear
CTA. I realize this was the product of a certain mindset that didn't
comprehend the value of brand marketing. And I certainly didn't
understand how direct response marketing almost inevitably comes into
conflict with the expectations around expertise.

I don't mean to portray direct response vs brand marketing as a binary
either/or decision. It's not. Even the ``Be all you can be'' campaign
had a direct response mechanism at the end. And you'll find other
elements of direct response marketing (like a focus on problems) in
brand marketing. I recently saw a Chromebook ad that used humor to
agitate the pain of not having computer backups. It had the tone of
brand marketing (the gift of a quick laugh) but the content of direct
response marketing (a focus on a problem, in this case someone dropping
their un-backed-up laptop computer).

Eventually most self-made experts will need to embrace the tools and
genre expectations of brand marketing.\# The Conflict Between Expertise
\& Direct Response Marketing

The genre expectations around valuable expertise conflict with direct
response marketing. In fact, if you're a genuine expert who uses direct
response marketing, you can undermine the perceived value of your own
expertise. Brand marketing is largely free from these kinds of
conflicts.

Let's explore why this is.

\hypertarget{authority}{%
\subsection{Authority}\label{authority}}

Authority is expertise plus trust. {[}\^{}I have my client Bob Lalasz to
thank for this elegant definition.{]} Subtract either expertise or trust
and you diminish authority.

At a certain point, direct response marketing corrodes trust. That is
the ``attack vector'' by which it undermines authority.

\hypertarget{expertise-and-genre-expectations}{%
\subsection{Expertise and Genre
Expectations}\label{expertise-and-genre-expectations}}

Our perception of experts is that they 1) do important stuff that takes
a lot of experience to do well or they 2) solve new important mysteries.
Or both.

In the case of \#1, expert marketing involves integrating with an
existing river of demand and existing institutionally-enabled discovery
mechanisms. {[}\^{}A currently well-known institutionally-enabled
discovery mechanism is TED talks. TED is the institution, and their
audience's respect for TED's curation makes TED a useful discovery
mechanism for interesting or compelling ideas.{]}

For \#2, the expert starts doing the work themselves and the
``marketing'' is sharing what they've done with others who get excited
about the expert's discoveries. This excitement leads to some form of
investment {[}\^{}The investment could be consulting work, direct
investment, or other forms of monetizing the expert's work of innovation
or discovery.{]} that sustains the expert's work.

There's an assumption at play here: experts would not invest in their
expertise if there was not an important market need supplying them with
plenty of demand for that expertise. Said more simply: saying you're an
expert implies there's lucrative demand for your expertise. Why else
would you have invested in it?

This assumption is why some who are \emph{not} genuine experts
\emph{claim} that they are experts. They are hoping to hack mental
heuristics to attract the kind of opportunity that genuine experts
enjoy. The problem happens when these folks use direct response
marketing. Their self-labeling as an expert sends one signal (``I have
lots of naturally-occurring opportunity'') while their marketing sends a
different signal (``I need to use inexpensive, efficient marketing to
capture every dribble of opportunity that might be out there, including
opportunity that I manufacture through my marketing.'')

A second assumption is at play in the genre expectations around
expertise: the importance of the expertise is self-evident. Prospects
will get exposed to the work and naturally respond because the
importance is self-evident.

This genre assumption sets up another conflict. If the supposed expert
is using a form of marketing that works very hard to highlight problems
and pains and to portray the expert as a solution for those
problems/pains, are they \emph{really} an expert? In other words, if
they were really an expert, wouldn't it be obvious that they are?

\hypertarget{signals}{%
\subsection{Signals}\label{signals}}

Direct response marketing signals a need to sell something we wouldn't
already buy on our own. This reads as unimportance (of the product) or
neediness (of the seller).

Direct response marketing suggests we are manufacturing an artificial
sort of demand.

Direct response marketing also signals that the business is focused on
short term transactional relationships {[}\^{}Hat tip Frank McClung for
surfacing this in a helpful discussion{]}. Direct response marketing
often uses a tone of pressure, urgency, and even fear to push the
recipient towards responding or buying.

Our experience with experts elsewhere in life is that we have to seek
them out, and once we find them they are objective in applying their
expertise, and throughout the exchange there's the assumption that the
matter at hand is \emph{actually} important and so we will act
accordingly and we won't need the expert to coach or motivate us to take
action. The way direct response marketing uses external pressure to
encourage action -- a short-term, transactional stance -- is in direct
conflict with the normal assumptions around genuine expertise.

Do we want short term/transactional expertise? Sometimes, but generally
we do not.

\hypertarget{practical-issues}{%
\subsection{Practical Issues}\label{practical-issues}}

The timing, volume, and cadence with which direct response marketing is
often delivered is at odds with the way senior managers deploy their
time and attention. Colloquially: they're too busy for this shit.

Direct response marketing timeline expectations are also often out of
step with the reality of the kind of people who would be economic buyers
of world-class expertise. Colloquially: they're focused on big,
important, long-term shit, not short-term transactional troubleshooting.

Senior managers have a multitude of concerns, but two are relevant here.
1) Time efficiency and 2) conservation of status {[}\^{}``Conservation
of status'' means generally avoiding doing stuff that diminishes status
and authority, and that means delegating certain tasks that are
``beneath'' the manager's position{]}. Both of these concerns make it
less likely a senior manager will pay attention to direct response
marketing. If they face an important decision, they are more likely to
delegate what they consider lower-level research, get referrals from
peers, or attend an event meant for them and their peers.

A heavy, high-pressure direct response marketing response tone amplifies
all of these conflicts between direct response marketing and the promise
of expertise.

\hypertarget{why-brand-marketing-is-largely-free-of-these-conflicts}{%
\subsection{Why Brand Marketing Is Largely Free Of These
Conflicts}\label{why-brand-marketing-is-largely-free-of-these-conflicts}}

Brand marketing is largely free of conflicts between our genre
expectations of experts and the realities of brand marketing. Why?

First, a posture of (perhaps extreme) generosity is more compatible with
the genre expectations around expertise. Remember our earlier assumption
that some experts are solving important mysteries? They share what
they've learned in the course of their ``detective work'' in order to
attract future investment in that work. Experts doing this repeatedly
over time trains us to expect experts to share.

A mode of sharing that uses more ``expensive media'' {[}\^{}Expensive
media might be speaking that involves travel, an extremely well-produced
podcast, a regular column in a high-status publication, and so on.{]} is
more compatible with the genre expectations around expertise. We expect
experts to be knowledgeable enough about the culture they seek to impact
(remember cultural awareness is a hallmark of brand marketing) to choose
institutions, venues, and publications that can help them create maximum
impact.

Finally, brand marketing builds trust in a more relaxed way than direct
response marketing. Brand marketing usually integrates with cultural
institutions that are already trusted, gives gifts that are more
genuinely generous, and takes potentially costly risks that direct
response marketers avoid.

\hypertarget{the-bootstrapping-gap}{%
\subsection{The Bootstrapping Gap}\label{the-bootstrapping-gap}}

There's a good match between the genre expectations around expertise and
the realities of brand marketing. Except for this one really troublesome
area: stuff that's important, but the market awareness isn't there yet.

\hypertarget{market-awareness}{%
\subsubsection{Market Awareness}\label{market-awareness}}

If we take smartphone penetration as a proxy for awareness of the
usefulness of smartphones, in 2007 only 7.8\% of the current market
awareness around smartphones existed. Colloquially, 92.2\% of the market
for mobile phones in 2007 was \emph{not} aware of the utility and value
of a smart mobile phone. We might call this an un-aware market. A new
thing exists, but the vast majority of the market isn't aware of the
value of or need for this thing. It was quickly apparent to some people
that smartphones had the potential to change \emph{everything}. And they
did in fact change so many aspects of modern life.

I routinely work with clients who face this same gap between the
potential of their expertise to drive significant, sector-wide change
for the better and an un-aware market. Additionally, these experts are
generally in a bootstrapping phase.

An unaware market {[}\^{}Calling a market unaware isn't a criticism,
it's merely a way of characterizing the market's relationship to some
new thing they're largely unaware of.{]} is also one where the market's
institutions aren't likely to be ready or eager to amplify a message
about the new thing. Unless the institution is used to finding and
amplifying these kinds of messages. The TED organization is one such
institution.

Real but unproven expertise + unaware market + cautious institutions;
how do you use brand marketing in this situation? There are 4 potential
ways:

\begin{enumerate}
\def\labelenumi{\arabic{enumi}.}
\tightlist
\item
  Walk your way up the ``ladder'' of institutions, starting with lower
  status ones with less stringent curation. Your message won't change
  much along the way, but you'll pay your dues for a while to earn the
  institutional acceptance and support you want to amplify your message.
\item
  Attack the margins. Blend a message an un-aware market can accept with
  your message about change or innovation. Titrate up on the
  innovation/change message as you can.
\item
  Use direct response marketing to gain momentum and build market
  awareness. In other words, bypass the culture's institutions and go
  direct to members of the culture using the tools of direct response
  marketing.
\item
  Build your own institution. You'll know if this is for you. Honestly,
  it probably isn't, since it's a formidable, complex, long-term play.
  {[}\^{}I'm not saying you're not capable of this. I am saying
  institution-building might not be the best path to your success as a
  self-made expert.{]}
\end{enumerate}

\hypertarget{summary-2}{%
\subsection{Summary}\label{summary-2}}

Genuine expertise eventually conflicts with direct response marketing.
Because direct response marketing signals need and manipulation, it
undermines the trust that is so critical to the authority experts need
to create impact.\# Why Do We Use Direct Response Marketing?

If direct response marketing eventually undermines our authority, why do
we use it at all?

You'll recognize some of these themes from the previous chapter, but
here we'll drill into why we make a short-term, utilitarian choice that
won't serve us our entire career.

\hypertarget{bootstrapping-1}{%
\subsection{Bootstrapping}\label{bootstrapping-1}}

Self-made expertise is a journey from implementor to advisor. This
journey happens outside the support systems of academia or the licensed
professions. Self-made experts build up credibility, access, and
expertise on their own in an entrepreneurial fashion.

This bootstrapping phase is a demanding time filled with hard, risky
work and careful tradeoffs. One such tradeoff is between brand
marketing, which poses no real conflict with the expert's authority, and
direct response marketing, which does.

During this bootstrapping phase, we need some things more than we need
the benefits of brand marketing. Those include:

\hypertarget{efficiency}{%
\subsubsection{Efficiency}\label{efficiency}}

Direct response marketing is more efficient than brand marketing, and
during the bootstrapping phase we need the leverage that comes from this
efficiency.

To be accurate, direct response marketing does not efficiently protect
the expert's authority. But during the bootstrapping phase, we need a
low-cost, effective way to reach those our growing expertise can help
more than we need efficient conservation of authority. Direct response
offers this leverage.

\hypertarget{ease-of-use}{%
\subsubsection{Ease of use}\label{ease-of-use}}

Direct response marketing is easier for most of us beginning marketers
to use than brand marketing. Assuming you're closer to the beginning
than end of a self-made expertise journey, I believe you'll find the
``d'' version of the following couplet easier to execute than the ``b''
version:

1.d: Write a useful article on how your expertise solves an expensive
problem for your clients, package it as a PDF, place it behind an opt-in
form on your website. 1.b: Prepare a talk on how your clients' industry
is changing and how your expertise paves a way forward for them, find a
national-level conference to deliver this talk at, pitch the conference
organizers, and do a good job of delivering the talk.

This is not the only example I could give you, but it pretty clearly
points out the relatively higher difficulty level inherent in brand
marketing.

\hypertarget{side-doors}{%
\subsubsection{Side doors}\label{side-doors}}

As we've established, the institutional platforms that brand marketers
use to distribute or amplify a message are often not available to those
of us bootstrapping expertise. At least not in the early stages of our
self-made expertise journey.

Institutional platforms are the front door.

Direct response marketing is a ``side door'' that gives you direct
access to an audience. Yes, you have to build this audience yourself.
And that's fine because the aforementioned efficiency of brand marketing
makes a smaller audience viable.

\hypertarget{default-status}{%
\subsection{Default status}\label{default-status}}

We use direct response marketing because it gives us these
bootstrapping-phase-friendly advantages. Efficiency, ease of use, and
access to a side door.

And finally, we use it because it's the default.

I just searched Google for ``marketing for consultants''. Here's the
advice from the first page of search results:

\begin{itemize}
\tightlist
\item
  \url{https://www.consultingsuccess.com/10-proven-marketing-tactics-for-consultants-and-coaches}

  \begin{itemize}
  \tightlist
  \item
    10 coaches/consultants asked about marketing that works for them, 7
    answer in terms of direct response marketing.
  \end{itemize}
\item
  \url{https://ducttapemarketing.com/7-steps-to-marketing-success-for-consultants-and-coaches/}

  \begin{itemize}
  \tightlist
  \item
    7 tips, 4 of which focus on direct response methods
  \end{itemize}
\item
  \url{https://www.entrepreneur.com/article/233303}

  \begin{itemize}
  \tightlist
  \item
    5 tips, 3 are focused on direct response
  \end{itemize}
\end{itemize}

That's not a cherry-picked list. I could keep going, but you get the
point. There's nothing wrong with direct response marketing being the
default. In fact, it might be for the best given the leverage it offers.

\hypertarget{summary-3}{%
\subsection{Summary}\label{summary-3}}

Direct response marketing's default position leads to a myopia about
brand marketing and makes it feel like, when the tension between direct
response marketing and expertise comes to a head, we're trapped in a
blind alley with no way out.

At the same time, brand marketing seems to be only for big companies
using mass market media. It's not. It just needs to be adapted for
self-made experts. There needs to be a defined migration path from
direct response to brand marketing. The next chapter covers that.\#
Managing The Transition

You'll decide one day to ``do marketing''. You'll almost certainly start
using the direct response marketing tools, operating within the genre
expectations of direct response.

And then you'll start noticing a tension between your authority as a
self-made expert and the direct response marketing you've been using.
This tension may reveal itself gradually and subtly, or a prospect may
give you the gift of a frank wakeup call conversation where they
spotlight it.

You'll resolve to change how you connect and build trust with prospects.
What then?

\hypertarget{your-status-quo}{%
\subsection{Your Status Quo}\label{your-status-quo}}

I can't know exactly what \emph{your} marketing system will look like,
but I can tell you what a typical small consultant's direct response
marketing system tends to look like:

\begin{itemize}
\tightlist
\item
  Email course or lead magnet(s) that collect email addresses and add
  those to a list. In other words, gated content assets.
\item
  A way of promoting gated content assets. Could be paid ad traffic, but
  more likely to be scrappy, inexpensive activities like podcast
  guesting, guest posting on other sites, SEO, etc.
\item
  Regular email marketing to the list.
\item
  A way of identifying email list members or other leads who are ready
  to act quickly. Could be a ``free consultation'', introductory
  diagnostic service, a survey delivered at some point, or regular calls
  to action that are meant to identify these fast-acting leads.
\item
  A philosophy of using CTAs that effects the design of your site. In
  other words, a site that is pretty heavy on calls to action.
\item
  A tone in your communication that focuses on problems, perhaps in a
  somewhat heavy-handed way.
\item
  Possibly: products or other digital assets that are part of a ``ladder
  of products and services''. These products possibly use engineered
  pricing with 3 price tiers, etc.
\item
  Other digital detritus like landing pages, etc.
\end{itemize}

\hypertarget{about-the-transition}{%
\subsection{About The Transition}\label{about-the-transition}}

I'd like to be able to end this chapter now with the admonition that the
transition is as simple as changing how you deploy CTAs, or simply
dropping CTAs altogether. It might start with that, but I don't think
it's as simple as that.

Some marketing is ephemeral, and some is not. Your investment in direct
response marketing will leave a ``residue'' of various assets and the
intertia of doing things a certain way.

If you decide to do things differently, that residue and inertia doesn't
just go away. You'll have cleanup to do.

Furthermore -- and you can trust me on this because I've seen the
firsthand consequences and pain of ignoring this advice in my own
business -- this is a transition rather than an abrupt switchover. If
you treat it as the latter, expect an interruption in lead generation,
or expect to begin a rapid switchover and see it turn into a
slow-motion, delayed, disruptive switchover because life happened at an
inopportune moment.

\hypertarget{practical-stuff}{%
\subsection{Practical stuff:}\label{practical-stuff}}

Then what? Again, that's why I wanted to write this. So you know what to
do when direct response starts to become a poor fit for your business.

The first thing to do is to take stock of where you are. How exactly
might the tone of your marketing be a tone of pressure, for example?

The second thing to do is to resolve to avoid feeling shame or
disappointment at how you've done marketing in the past. Those emotions
don't help, and whatever you've done with direct response marketing
might be what's enabled you to arrive at this privileged moment of
considering a style of marketing that's less efficient.

When that moment comes, and you sense you need to transition from direct
response to brand marketing, what specifically do you do?

You might be able to strip out gates around content assets. That might
be enough.

You might be able to change the tone of what you're already doing. The
tone might have been fine and not need changing or, like I had to, you
might need to alter it at a more fundamental level.

You might be able to consolidate a constellation of gated, small-scale
direct response assets into fewer more impactful ungated gifts. As a
side benefit, you might be able to feel less stressed when you crack
open the dashboard of your marketing automation software. :) You might
have fewer automated email sequences to maintain, for example.

And there might be other easy things you can do, like dropping
engineered pricing for products and going with more genre-appropriate
price structures, for example.

So there might be some easy, more tactical changes you can make. And
then after making those changes, things might get \emph{way more
difficult} because you're ditching a crutch that you came to rely on,
and now you need to further develop a new capability, which we could
summarize thusly:

\textbf{Be able to get on a stage for 45 minutes with minimal or no
props and no CTA at all and say something so impactful that -- assuming
it's the right audience -- at least one person approaches you after the
talk to ask for a meeting to explore working together.}

Your brand marketing may not make heavy use of speaking. That's fine.
But if you think of the ``impactful no-CTA talk'' as a sort of litmus
test, you'll be led to organize your thinking and points of view such
that brand marketing will be a workable tool for you, and you'll be able
to resolve that direct response-expertise conflict by lightening up on
the direct response tools as much as your audience needs you to.

Another way to think of this\ldots{}

Q: When you can let go of direct response tools and methods?

A: When you can get on a stage for 45 minutes with minimal or no props
and no CTA at all and say something so impactful that -- assuming it's
the right audience -- at least one person approaches you after the talk
to ask for a meeting to explore working together.

You can think of it like earning your way out of direct response land.

\hypertarget{summary-4}{%
\subsection{Summary}\label{summary-4}}

You might not do the no-CTA talk. The specifics of how you do it might
be different. For example, you might do a no-CTA \emph{book} that has
similar dynamics to the talk.

But the idea is the same: you'll have a message that's so powerful that
it drowns out the inefficiencies of brand marketing; it overwhelms those
inefficiencies with such \emph{substance} that the means of marketing
becomes transparent and the power of the message is the only thing your
prospects notice.\# Coda

My not-at-all secret agenda is to support the cultivation of self-made
expertise.

I hope this guidebook has given you language and a conceptual framework
that helps you understand what might feel icky about direct response
marketing, and a map out of that territory. I hope this framework and
map amplify your focus on cultivating ever more valuable self-made
expertise, which is the ultimate fuel for moving from direct response to
brand marketing.

\hypertarget{parting-advice}{%
\subsection{Parting advice}\label{parting-advice}}

I've tried to paint direct response and brand marketing with a fair
brush. The tools are neutral; it's all in how you use them. There are
experts I respect who use direct response marketing, but with a tone and
style that gives them useful leverage but without excessively
undermining their authority.

That said, if you feel disaffected with direct response, use that
emotion to fuel a transition to brand marketing! Again, it's a
transition, not a sudden state change.

As you do so, the thrifty direct response mindset can serve you well in
the early stages of brand marketing as you re-use direct response assets
but with a brand marketing mindset.

I hope having a model for thinking about all of this helps make sense of
the world of marketing. Don't let it intimidate you!

Progress is the new perfection.



\end{document}
